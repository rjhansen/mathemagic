\documentclass[10pt,letterpaper,oneside,notitlepage]{article}
\usepackage{textcomp}
\usepackage[x11names]{xcolor}
\usepackage{mathtools}
\usepackage{concmath}
\usepackage{merriweather}
%\usepackage{eulervm}
\usepackage{tabularray}
\usepackage{hyperref}
\usepackage{colortbl}
\usepackage{fancyhdr}
\pagestyle{fancy}
\hypersetup{
    colorlinks = true,
    allcolors = blue
}
\usepackage{url}

\begin{document}
\title{A Little Mathemagic}
\author{Rob Hansen \href{mailto:rob@hansen.engineering?subject=A\%20Little\%20Mathemagic}{\textlangle rob@hansen.engineering\textrangle}}
\maketitle
\tableofcontents
\begin{abstract}
  \noindent
  The question was posed, ``do things really get heavier as they move
faster?'' The answer---``not really, but it sure looks that way''---is a
little easier to understand with some math.
\end{abstract}

\section{Preface}
\subsection{Dedication and thanks}
For Nick and Alex, Silas, Milo and Margot. You're too young to be dealing
with physics or algebra yet, but when the time comes I hope your parents
share this and tell you I was so weird I wrote this \textit{for fun.}

Thanks to \href{https://startswithabang.com}{Dr.~Ethan Siegel}
\footnote{\textit{https://startswithabang.com}} for the
initial inspiration to write this.

\subsection{Legal notices}
You may use and share this work under the terms of the 
\href{https://creativecommons.org/licenses/by-nd/4.0/}{Creative Commons Attribution-NoDerivatives 4.0 International}
\footnote{\textit{https://creativecommons.org/licenses/by-nd/4.0/}} license.

\subsection{Getting the latest and reporting errors}
This is a living document. You can always find the latest release at 
\href{https://github.com/rjhansen/mathemagic/releases}{its GitHub page}.
\footnote{\textit{https://github.com/rjhansen/mathemagic/releases}}

I've taken all reasonable care in preparing this, but errors sometimes
make it through. If you find one, please
\href{https://github.com/rjhansen/mathemagic/issues}{let me know}.
\footnote{\textit{https://github.com/rjhansen/mathemagic/issues}}

\section{Where Things Stand}
When I was in undergrad, we learned that as an object gained
velocity it also gained mass. There was a special word given to this mass,
namely \textit{relativistic mass,} to separate it from the concept of the
mass of the same object at rest, or \textit{rest mass.}

Neither term is in good form any more. Instead, all mass is rest mass,
and ``relativistic mass'' is better understood to be just ``how velocity
affects an observer's view of things.''

\subsection{Why was the old way a problem?}

It raised many more questions than answers. If more mass was being created,
well, from where? Out of what? If you had a just-barely subcritical mass
of $^{239}_{94}\text{Pu}$ and gave it to Fernando Valenzuela, could he throw it fast
enough to trigger an atomic explosion? Were there additional elementary
particles being created? How were they structured? Did this mean two
observers, one at speed and one at rest, would disagree about how many
moles of a chemical they each saw?

That last one was a real nail-biter. Relativity was already demanding so
much of physicists: the inability to agree on simultaneity, distance, time,
mass, and more. Were we going to have to add even \textit{counting} to that?

\subsection{How did we resolve it?}

By realizing we didn't need relativistic mass at all. Instead, we now
realize that velocity doesn't add mass: it simply amplifies the way
mass interacts with the cosmos. And we did this with nothing more than
$E = mc^{2}$!

Well, kind of.

\section{The details}

Einstein didn't equate mass with energy: he equated mass \textit{and
momentum} with energy. The total energy of a system isn't given only by
its mass, but also by how much energy it's storing in its motion
(kinetic energy). The actual, full version of the mass-momentum-energy
equation is:

\[
E^{2} = m^{2}c^{4} + p^{2}c^{2}
\]

From just this one equation, we can see how apparent mass changes with
velocity. Let's dive in!

\subsection{The derivation}

\begin{longtblr}{
  colspec = { c p{5cm} p{5cm} },
  row{even} = {Ivory2},
  width = \linewidth,
  rowhead = 1
}
  Step & Equation & Justification \\
  1 & \[ E^{2} = m^{2}c^{4} + p^{2}c^{2} \] & Always begin derivations by restating
  the information given. In this case, since we were given Einstein's
  equation for the mass-momentum-energy equivalence, we write that down. \\
  2 & \[ \pm E = \sqrt{m^{2}c^{4} + p^{2}c^{2}} \] & Take the square root of
  both sides\\
  3 & \[ \sqrt{m^{2}c^{4} + p^{2}c^{2}} \] & We're going to omit the left-hand
  side going forward, as it's not going to undergo any changes.\\
  4 & \[ \sqrt{m^{2}c^{4} + m^{2}\gamma^{2}v^{2}c^{2}} \] & Since in relativity
  momentum ($p$) is mass ($m$), the Lorentz factor ($\gamma$), and velocity ($v$),
  we replace $p$ with $m\gamma v$. Following the normal rules of algebra, 
  $p^{2} = m^{2}\gamma^{2}v^{2}$.\\
  5 & \[ \gamma = \frac{1}{\sqrt{1 - \left(\frac{v}{c}\right)^{2}}}\] & This is
  the Lorentz factor. We'll make use of this later. \\
  6 & \[ \gamma = \frac{1}{\sqrt{1 - \beta^{2}}}\] & $\frac{v}{c}$, or ``just
  what fraction of lightspeed are you traveling?'', comes up so
  frequently in physics it's usually just referred to as $\beta$.\\
  7 & \[ \sqrt{m^{2}c^{2}\left(c^{2} + \gamma^{2}v^{2}\right)} \] & Since both our terms
  under the square root sign have an $m^{2}c^{2}$, we can simplify things a
  little bit more by moving it out front as a shared factor.\\
  8 & \[ \sqrt{m^{2}c^{4}\left(1 + \frac{\gamma^{2}v^{2}}{c^{2}}\right)} \] & Dividing the
  second term by $c^{2}$ gives us another $c^{2}$ term to collect out
  front.\\
  9 & \[ \sqrt{m^{2}c^{4}\left(1 + \gamma^{2}\beta^{2}\right)}\] & \ldots and
  look, we can replace the $\frac{v^{2}}{c^{2}}$ with $\beta^{2}$.\\
  10 & \[ \sqrt{m^{2}c^{4}\left(1 + \beta^{2}\gamma^{2}\right)}\] &
  Reformat the $\beta$-term for ease of reading. ($\beta$ comes before $\gamma$
  in the Greek alphabet, and just as an $xy$ term looks more natural to us than
  $yx$, so too do Greek-symboled equations look better if they're in order.)\\
  11 & \[ \sqrt{m^{2}c^{4}}\cdot\sqrt{1 + \beta^{2}\gamma^{2}} \] &
  $\sqrt{ab} = \sqrt{a}\cdot\sqrt{b}$\\
  12 & \[ \pm mc^{2}\sqrt{1 + \beta^{2}\gamma^{2}} \] & Since 
  $\sqrt{m^{2}c^{4}} = \pm mc^{2}$, we simplify and substitute.\\
  13 & \[ \gamma = \frac{1}{\sqrt{1 - \beta^2}}\] & Restatement of \#6\\
  14 & \[ \gamma^2 = \frac{1}{1 - \beta^2}\] & Simplify\\
  15 & \[ \beta^{2}\gamma^{2} = \frac{\beta^2}{1 - \beta^2}\] & Express our Lorentz term
  entirely in terms of ``just what fraction of lightspeed is it going?''\\
  16 & \[ \pm mc^{2}\sqrt{1 + \frac{\beta^2}{1 - \beta^2}} \] & \#12, simplified\\
  17 & \[ \pm E = \pm mc^{2}\sqrt{1 + \frac{\beta^2}{1 - \beta^2}} \] & 
  Re-introduce our long-ignored left-hand side\\
  18 & \[ E = mc^{2}\sqrt{1 + \frac{\beta^2}{1 - \beta^2}} \] & Restrict ourselves
  to positive-mass, positive-energy solutions\\
  19 & \[ E = mc^{2} \] & At low speeds, $\beta\approx0$. $\sqrt{1 + \frac{0^2}{1 - 0^2}}$
  is going to be ridiculously close to $1$, meaning that massive particles at rest obey
  $E = mc^2$.\\
\end{longtblr}

\subsection{Notes on the derivation}

Although each mathematical step is fully presented and justified, there
are some physics issues that come up in the doing. They're discussed here.

\begin{enumerate}
  \setcounter{enumi}{1}
  \item
  Every mathematician will tell you that square roots computed over the
  positive reals always produce two answers, one positive and one negative.
  This causes mathematicians to wonder, ``that's odd: do negative energies
  actually exist?'' Physicists have generally decided they're much 
  happier ignoring the implications of such rigor.
  \setcounter{enumi}{3}
  \item
  Well, almost. This is a correct statement for the momentum of things
  that have mass. But massless particles $\left(m = 0\right)$ like photons also
  have energy, which means they also have momentum. It's just that we
  have to compute it differently: photon momentum, for instance, is
  $p = \frac{h}{\lambda}$, where $h$ is Planck's constant and $\lambda$
  the wavelength. (Or, equivalently, $p = hf$, where $f$ is its
  frequency.)
  \setcounter{enumi}{7}
  \item
  It's okay if you don't immediately see why this is a
  big win---it's the kind of thing you develop a sort of mathematical
  intuition about. Your sense of smell will come along as you do more
  algebra.
  \setcounter{enumi}{16}
  \item
  If you've ever wondered where speculative physics gets
  the idea for negative energy, the answer is right here! Einstein's
  theory of relativity clearly allows for the possibility of negative
  energy, as well as negative mass. Of course, just because something
  is permitted by our theories doesn't mean the cosmos actually does
  it\ldots
  \item
  The physics community has an
  odd take on the weirder bits of Einstein. If you openly state your 
  paper is basically science fiction, you can talk about negative masses
  and energies all you want. On the other hand, if you have the temerity 
  to try submitting a negative mass or energy condition in a Real Physics
  Paper they'll laugh you out of town. So, with reluctance, we erase the
  $\pm$ symbol and arbitrarily, and without evidence, declare the universe
  to only permit positive solutions.
\end{enumerate}

\section{Conclusion}
According to Einstein's full mass-momentum-energy equation, the energy
of a body in motion is proportional to its mass, the square of the speed
of light, and a square-root term that all hinges around how fast the body
is moving. We don't need to introduce bizarre notions of ``relativistic
mass'' to explain the additional energy picked up by objects as they
travel: that's fully explained just by the presence of the final term.

We now have answers to all our thorny questions from earlier.

\begin{itemize}
  \item
  \textbf{If more mass is being created, well, from where? Out of what?}
  No new mass is created.

  \item
  \textbf{Could a Hall of Fame baseball pitcher throw a baseball made 
  of plutonium fast enough that it would become a critical mass and explode?}
  No, thankfully.

  \item
  \textbf{Can two observers agree on how many moles of a chemical they see?}
  Sure.
\end{itemize}

\end{document}
